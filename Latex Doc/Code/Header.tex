\documentclass[11pt, a4paper, onecolumn,  abstracton, bibliography=totoc]{scrartcl} 

\usepackage{silence}
\WarningFilter{titlesec}{Non standard sectioning command}
%%%%%%%%%%%%%%%%%%%% Sprache & Zeichen %%%%%%%%%%%%%%%%%%%%%
\usepackage[english]{babel}    %Englisch und Deutsch
\usepackage{lmodern}    %Scharfe Schrift
\usepackage[T1]{fontenc}    %T1 als Zeichensatz mit Umlaute für         pdf
\usepackage[utf8x]{inputenc}    %für Umlaute
\usepackage[output-decimal-marker={,},exponent-product=\cdot]{siunitx} %einfache Darstellung von Einheiten z.B:             
    %\SI{-273,15}{\degree\celsius} oder 
    %15~\si{\newton\per\square\meter}
	\sisetup{locale = DE , per-mode = symbol}   %Einstellungen     dazu
\usepackage{csquotes}   %gute Darstellung der                             Anführungszeichen
\usepackage{fancyhdr}   %Fußzeilen
\usepackage{titletoc}   %Darstellung von Überschriften im                 Inhaltsverzeichnis
\usepackage{amsmath} %verschiedene                     Gleichungsumgebungen erlaubt Integrale
\usepackage{textcomp}   %spez. Zeichen wie copyright, Musiknote
\usepackage{pxfonts}    %griechisches Alphabet
\usepackage{latexsym}   %ein paar sehr specialige Mathesymbole
\usepackage{amssymb}    %diverse gebräuchliche Mathesymbole
%\usepackage[version=3]{mhchem}	%chemische Symbole
\usepackage{textcomp}   %Euro-Zeichen mit \texteuro
\usepackage[colorlinks, pdfpagelabels,                             pdfstartview=FitH,bookmarksopen=true,                          bookmarksnumbered=true,linkcolor=black, plainpages=false,      hypertexnames=false, citecolor=black,                          urlcolor=black]{hyperref} %erlaubt in Tabellen den             Spaltentyp S dann werden alle Zahlen am Dezimaltrennzeichen     ausgerichtet, Prozent -> Prozentzeichen
\usepackage[ddmmyyyy, nodayofweek]{datetime}
%%%%%%%%%%%%%%%%%%%%%%%% Literatur %%%%%%%%%%%%%%%%%%%%%%%%%
%\usepackage{chapterbib}    %Bib je Kapitel
\usepackage{bibgerm}    %deutsches Literaturverz.
\usepackage[round]{natbib}  %Benutzerdefiniertes Zitieren
\usepackage{cite}   %verbessertes Zitieren
\bibliographystyle{apalike}		%deutsche zitierweise
\setcitestyle{authoryear,open={(}, close={)}} %finetuning lit.     Verzeichnis nach Dirks
\usepackage{multibib} % ermöglicht die Verwendung mehrerer         BibTeX-Dateien und mehrerer Literaturverzeichnisse 
\newcites{others}{Other Sources}   %referiert auf         Sonstige Quellen bei \citesonstige{...}
\bibliographystyleothers{apalike}  %Zitierstil von     neuer Bib 
\newcites{patents}{Patents and Norms}   %referiert auf         Sonstige Quellen bei \citesonstige{...}
\bibliographystylepatents{apalike}  %Zitierstil von     neuer Bib 

%%%%%%%%%%%%%%%%%%%%%%%% Abb./Tab. %%%%%%%%%%%%%%%%%%%%%%%%%
\usepackage{graphicx}   %Einbindung von Abb. möglich
\usepackage{wrapfig}    %Figures von Text umflossen
\usepackage[font={footnotesize}]{caption} %Unterschrift,        veränderliches Aussehen in der ersten Klammer
\usepackage{float}  %Stelle für Figure [..] spezifiziert
\usepackage{subcaption} %Unterunterschrift
%\usepackage{subfigure} %Unterschrift für versch. Figures
\usepackage{booktabs}   %horizontale Linien in Tabelle
\usepackage{multirow}   %Zelle in Tabelle von mehreren Zeilen
%\usepackage{slashbox}   %Zelle in Tabelle durch Diagonale         getrennt
\usepackage{diagbox}    %Zelle in Tabelle durch Diagonale         getrennt
\usepackage{pdfpages}
\usepackage{overpic}
\usepackage{svg}
%%%%%%%%%%%%%%%%%%%%%% Außerliches %%%%%%%%%%%%%%%%%%%%%%%%%
\usepackage[a4paper, left=2.5cm, right=2.5cm, top=2.5cm, bottom=2.5cm]{geometry}   %Seitenränder
\usepackage[onehalfspacing]{setspace} %Zeilenabstand
\usepackage{nameref}    %Querverweise
\usepackage{comment}    %Überspring Absätze wie zB Kommentare
\usepackage{ragged2e}   %erlaubt links-/rechtsbündig etc.
\usepackage{pdflscape} %landscape für Querformat
\usepackage{url}    %ermöglicht Trennen an best. Zeichen
%\usepackage[table,xcdraw]{xcolor} %Veränderung von Schriftfarbe
\usepackage[page,header]{appendix}  %Anhang
\usepackage{titlesec}
\usepackage{tocloft}
\usepackage{todonotes}
\usepackage{mathtools}

\usepackage{tikz} % Required for drawing custom shapes
\usetikzlibrary{arrows}
\usetikzlibrary{math}
\usetikzlibrary{calc}
\usetikzlibrary{angles,quotes}
\usetikzlibrary{shapes, decorations}
\usepackage{tkz-euclide}
%%%%%%%%%%%%%%%%% Veränderung von Befehlen %%%%%%%%%%%%%%%%%%%%
\setlength{\parindent}{0pt} %% entfernt Einrückung 
%\newcommand{\Abkürzung}{Abkürzung für ...} %Kurzbefehle
%\renewcommand{\figureautorefname}{Abb.}  %Veränderung von                                vorhandenen Bildunterschirften
%\renewcommand{\tableautorefname}{Tab.}
%\setlength{\footskip}{30pt}
\renewcommand{\cftsecfont}{\scshape\fontfamily{phv}\selectfont}

\renewcommand{\cftsubsecfont}{\scshape\fontfamily{phv}\selectfont}
\renewcommand{\cftsubsecpagefont}{\scshape\fontfamily{phv}\selectfont}
\renewcommand{\cftsubsubsecfont}{\scshape\fontfamily{phv}\selectfont}
\renewcommand{\cftsubsubsecpagefont}{\scshape\fontfamily{phv}\selectfont}


\renewcommand{\cftpartfont}{\scshape\fontfamily{phv}\selectfont}
\renewcommand{\cftpartpagefont}{\scshape\fontfamily{phv}\selectfont}

\renewcommand{\cftloftitlefont}{\Large\scshape\fontfamily{phv}\selectfont}
\renewcommand{\cftfigfont}{\large\scshape\fontfamily{phv}\selectfont}
\renewcommand{\cftfigpagefont}{\scshape\fontfamily{phv}\selectfont}

\renewcommand{\cftlottitlefont}{\Large\scshape\fontfamily{phv}\selectfont}
\renewcommand{\cfttabfont}{\large\scshape\fontfamily{phv}\selectfont}
\renewcommand{\cfttabpagefont}{\scshape\fontfamily{phv}\selectfont}

\renewcommand{\cftsecaftersnum}{.}

\newcommand{\HR}[1]{{\color{HR} #1}}
\definecolor{HR}{HTML}{ffa233}

\urlstyle{own}

\makeatletter
\def\url@ownstyle{%
  \@ifundefined{selectfont}{\def\UrlFont{\sf}}{\def\UrlFont{\small\ttfamily}}}

\titleformat{\section} % command
    [block] % shape
    {\Large\scshape\fontfamily{phv}\selectfont} % format
    {\thesection.} % label
    {1ex} % sep
    {} % before-code
    [\vspace{-2ex} \rule{\textwidth}{0.3pt}] % after-code
\titleformat{\subsection} % command
    [block] % shape
    {\large\scshape\fontfamily{phv}\selectfont} % format
    {\thesubsection} % label
    {1ex} % sep
    {} % before-code
    [\vspace{-2ex} \rule{\textwidth}{0.3pt} \vspace{-4.5ex}] % after-code
\titleformat{\subsubsection} % command
    [block] % shape
    {\large\scshape\fontfamily{phv}\selectfont} % format
    {\thesubsubsection} % label
    {1ex} % sep
    {} % before-code
    [\vspace{-2ex} \rule{\textwidth}{0.3pt} \vspace{-4.5ex}] % after-code
\titleformat{\part} % command
    [block] % shape
    {\LARGE\scshape\fontfamily{phv}\selectfont} % format
    {} % label
    {1ex} % sep
    {} % before-code
    [\vspace{-2ex} \rule{\textwidth}{0.3pt}\vspace{-2.7ex} \rule{\textwidth}{0.3pt}] % after-code
\makeatother

