
\documentclass[11pt, a4paper, onecolumn,  abstracton, bibliography=totoc]{scrartcl} 

\usepackage{silence}
\WarningFilter{titlesec}{Non standard sectioning command}
%%%%%%%%%%%%%%%%%%%% Sprache & Zeichen %%%%%%%%%%%%%%%%%%%%%
\usepackage[english]{babel}    %Englisch und Deutsch
\usepackage{lmodern}    %Scharfe Schrift
\usepackage[T1]{fontenc}    %T1 als Zeichensatz mit Umlaute für         pdf
\usepackage[utf8x]{inputenc}    %für Umlaute
\usepackage[output-decimal-marker={,},exponent-product=\cdot]{siunitx} %einfache Darstellung von Einheiten z.B:             
    %\SI{-273,15}{\degree\celsius} oder 
    %15~\si{\newton\per\square\meter}
	\sisetup{locale = DE , per-mode = symbol}   %Einstellungen     dazu
\usepackage{csquotes}   %gute Darstellung der                             Anführungszeichen
\usepackage{fancyhdr}   %Fußzeilen
\usepackage{titletoc}   %Darstellung von Überschriften im                 Inhaltsverzeichnis
\usepackage{amsmath} %verschiedene                     Gleichungsumgebungen erlaubt Integrale
\usepackage{textcomp}   %spez. Zeichen wie copyright, Musiknote
\usepackage{pxfonts}    %griechisches Alphabet
\usepackage{latexsym}   %ein paar sehr specialige Mathesymbole
\usepackage{amssymb}    %diverse gebräuchliche Mathesymbole
%\usepackage[version=3]{mhchem}	%chemische Symbole
\usepackage{textcomp}   %Euro-Zeichen mit \texteuro
\usepackage[colorlinks, pdfpagelabels,                             pdfstartview=FitH,bookmarksopen=true,                          bookmarksnumbered=true,linkcolor=black, plainpages=false,      hypertexnames=false, citecolor=black,                          urlcolor=black]{hyperref} %erlaubt in Tabellen den             Spaltentyp S dann werden alle Zahlen am Dezimaltrennzeichen     ausgerichtet, Prozent -> Prozentzeichen
\usepackage[ddmmyyyy, nodayofweek]{datetime}
%%%%%%%%%%%%%%%%%%%%%%%% Literatur %%%%%%%%%%%%%%%%%%%%%%%%%
%\usepackage{chapterbib}    %Bib je Kapitel
\usepackage{bibgerm}    %deutsches Literaturverz.
\usepackage[round]{natbib}  %Benutzerdefiniertes Zitieren
\usepackage{cite}   %verbessertes Zitieren
\bibliographystyle{apalike}		%deutsche zitierweise
\setcitestyle{authoryear,open={(}, close={)}} %finetuning lit.     Verzeichnis nach Dirks
\usepackage{multibib} % ermöglicht die Verwendung mehrerer         BibTeX-Dateien und mehrerer Literaturverzeichnisse 
\newcites{others}{Other Sources}   %referiert auf         Sonstige Quellen bei \citesonstige{...}
\bibliographystyleothers{apalike}  %Zitierstil von     neuer Bib 
\newcites{patents}{Patents and Norms}   %referiert auf         Sonstige Quellen bei \citesonstige{...}
\bibliographystylepatents{apalike}  %Zitierstil von     neuer Bib 

%%%%%%%%%%%%%%%%%%%%%%%% Abb./Tab. %%%%%%%%%%%%%%%%%%%%%%%%%
\usepackage{graphicx}   %Einbindung von Abb. möglich
\usepackage{wrapfig}    %Figures von Text umflossen
\usepackage[font={footnotesize}]{caption} %Unterschrift,        veränderliches Aussehen in der ersten Klammer
\usepackage{float}  %Stelle für Figure [..] spezifiziert
\usepackage{subcaption} %Unterunterschrift
%\usepackage{subfigure} %Unterschrift für versch. Figures
\usepackage{booktabs}   %horizontale Linien in Tabelle
\usepackage{multirow}   %Zelle in Tabelle von mehreren Zeilen
%\usepackage{slashbox}   %Zelle in Tabelle durch Diagonale         getrennt
\usepackage{diagbox}    %Zelle in Tabelle durch Diagonale         getrennt
\usepackage{pdfpages}
\usepackage{overpic}
\usepackage{svg}
%%%%%%%%%%%%%%%%%%%%%% Außerliches %%%%%%%%%%%%%%%%%%%%%%%%%
\usepackage[a4paper, left=2.5cm, right=2.5cm, top=2.5cm, bottom=2.5cm]{geometry}   %Seitenränder
\usepackage[onehalfspacing]{setspace} %Zeilenabstand
\usepackage{nameref}    %Querverweise
\usepackage{comment}    %Überspring Absätze wie zB Kommentare
\usepackage{ragged2e}   %erlaubt links-/rechtsbündig etc.
\usepackage{pdflscape} %landscape für Querformat
\usepackage{url}    %ermöglicht Trennen an best. Zeichen
%\usepackage[table,xcdraw]{xcolor} %Veränderung von Schriftfarbe
\usepackage[page,header]{appendix}  %Anhang
\usepackage{titlesec}
\usepackage{tocloft}
\usepackage{todonotes}
\usepackage{mathtools}

\usepackage{tikz} % Required for drawing custom shapes
\usetikzlibrary{arrows}
\usetikzlibrary{math}
\usetikzlibrary{calc}
\usetikzlibrary{angles,quotes}
\usetikzlibrary{shapes, decorations}
\usepackage{tkz-euclide}
%%%%%%%%%%%%%%%%% Veränderung von Befehlen %%%%%%%%%%%%%%%%%%%%
\setlength{\parindent}{0pt} %% entfernt Einrückung 
%\newcommand{\Abkürzung}{Abkürzung für ...} %Kurzbefehle
%\renewcommand{\figureautorefname}{Abb.}  %Veränderung von                                vorhandenen Bildunterschirften
%\renewcommand{\tableautorefname}{Tab.}
%\setlength{\footskip}{30pt}
\renewcommand{\cftsecfont}{\scshape\fontfamily{phv}\selectfont}

\renewcommand{\cftsubsecfont}{\scshape\fontfamily{phv}\selectfont}
\renewcommand{\cftsubsecpagefont}{\scshape\fontfamily{phv}\selectfont}
\renewcommand{\cftsubsubsecfont}{\scshape\fontfamily{phv}\selectfont}
\renewcommand{\cftsubsubsecpagefont}{\scshape\fontfamily{phv}\selectfont}


\renewcommand{\cftpartfont}{\scshape\fontfamily{phv}\selectfont}
\renewcommand{\cftpartpagefont}{\scshape\fontfamily{phv}\selectfont}

\renewcommand{\cftloftitlefont}{\Large\scshape\fontfamily{phv}\selectfont}
\renewcommand{\cftfigfont}{\large\scshape\fontfamily{phv}\selectfont}
\renewcommand{\cftfigpagefont}{\scshape\fontfamily{phv}\selectfont}

\renewcommand{\cftlottitlefont}{\Large\scshape\fontfamily{phv}\selectfont}
\renewcommand{\cfttabfont}{\large\scshape\fontfamily{phv}\selectfont}
\renewcommand{\cfttabpagefont}{\scshape\fontfamily{phv}\selectfont}

\renewcommand{\cftsecaftersnum}{.}

\newcommand{\HR}[1]{{\color{HR} #1}}
\definecolor{HR}{HTML}{ffa233}

\urlstyle{own}

\makeatletter
\def\url@ownstyle{%
  \@ifundefined{selectfont}{\def\UrlFont{\sf}}{\def\UrlFont{\small\ttfamily}}}

\titleformat{\section} % command
    [block] % shape
    {\Large\scshape\fontfamily{phv}\selectfont} % format
    {\thesection.} % label
    {1ex} % sep
    {} % before-code
    [\vspace{-2ex} \rule{\textwidth}{0.3pt}] % after-code
\titleformat{\subsection} % command
    [block] % shape
    {\large\scshape\fontfamily{phv}\selectfont} % format
    {\thesubsection} % label
    {1ex} % sep
    {} % before-code
    [\vspace{-2ex} \rule{\textwidth}{0.3pt} \vspace{-4.5ex}] % after-code
\titleformat{\subsubsection} % command
    [block] % shape
    {\large\scshape\fontfamily{phv}\selectfont} % format
    {\thesubsubsection} % label
    {1ex} % sep
    {} % before-code
    [\vspace{-2ex} \rule{\textwidth}{0.3pt} \vspace{-4.5ex}] % after-code
\titleformat{\part} % command
    [block] % shape
    {\LARGE\scshape\fontfamily{phv}\selectfont} % format
    {} % label
    {1ex} % sep
    {} % before-code
    [\vspace{-2ex} \rule{\textwidth}{0.3pt}\vspace{-2.7ex} \rule{\textwidth}{0.3pt}] % after-code
\makeatother




%%%%%%%%%%%%%%%%%%%%%%%% Beginn %%%%%%%%%%%%%%%%%%%%%%%%%
\begin{document}
\fontfamily{ptm}\selectfont
\begin{titlepage}
\date{}

		
	\begin{figure}[h]\vspace{1 cm}
    \centering
    \includegraphics[width = 0.4\textwidth]{Pictures/HSRW_Logo.png}
    \end{figure}

	\begin{center}

	\large{
	Hochschule Rhein-Waal\\
	Marie-Curie-Strasse 1, 47533 Kleve\\
	Faculty Technology and Bionics\\
	Bionics M. Sc.\\}
	\vspace{2cm}
	
	\noindent\makebox[\linewidth]{\rule{\textwidth}{0.2pt}}
	\vspace{0.1cm}
	
	\begin{LARGE}\begin{bfseries} EVOLUTIONARY ALGORITHMS\\\end{bfseries} \end{LARGE}
	\vspace{0.3 cm}
\begin{LARGE}As a heuristic approach to the fair distribution of children in classes in primary schools\\ \end{LARGE}

\vspace{0.3cm}
		\noindent\makebox[\linewidth]{\rule{\textwidth}{0.2pt}}		

	
	\vspace{0.5cm}

		
\large{\textbf{\\Sophia Felicia Salome Döring}}
\large{{\\31519}}
\vspace{4.0cm} 
\end{center}

\vfill

\begin{tabular}{p{3cm}l}
    Module: & \\
    Teacher: & \\
    Submission Date: & \\
\end{tabular}

\thispagestyle{empty}



\end{titlepage}
\newpage
%%%%%%%%%%%%%%%%%%%%%%%%%%%%%%%%%%%%%%%%%%%%%%%%%%%%%%%%%%%%%%%%%%%%%%%
\section*{Table of Contents}
\startcontents[sections]
\printcontents[sections]{l}{1}{\setcounter{tocdepth}{3}}
\thispagestyle{empty}
\pagenumbering{arabic}
%%%%%%%%%%%%%%%%%%%%%%%%%%%%%%%%%%%%%%%%%%%%%%%%%%%%%%%%%%%%%%%%%%%%%%%
\newpage
\renewcommand{\listfigurename}{}
\section*{List of Figures}
\vspace{-6ex}
\listoffigures
\renewcommand{\listtablename}{}
\section*{List of Tables}
\vspace{-6ex}
\listoftables
\thispagestyle{empty}
%%%%%%%%%%%%%%%%%%%%%%%%%%%%%%%%%%%%%%%%%%%%%%%%%%%%%
\newpage
\begin{onehalfspace}

\section*{Abstract}
\newpage
\section{Introduction}
Equality, inclusion, and fairness of an educational system are of importance 
for the right to education. Even though this right and equal opportunity are 
stated in the german constitution (as well as in other constitutions), their 
implementation is complex and a controversial issue. \citep{zhang2014equality}

In order to realize the multifaceted challenge of equality, inclusion, and fairness
 (or to come as close as possible), a lot of work is put in in schools, especially at 
 the beginning of a child's school career, to find a fair and equitable class distribution,
  to achieve a pleasant class climate, to support children who need more support in different areas, 
  and to create a familiar environment, e.g. through friends from preschool/kindergarten.

But the controversy already starts with the question of what conditions, characteristics, 
and abilities of children are crucial for class formation. Depending on the school's 
philosophy, country, and other regulations, not only teachers, but also the parents can take 
part in the decision, especially if there are classes for specialization. Therefore, teachers 
spent a significant amount of time finding a fitting and fair class distribution before the beginning
 of the school year. 
\\

 In the German public school system, a school entrance examination is conducted to assess whether 
 a child is ready to start school. This examination evaluates not only child's school readiness but also 
 a child's cognitive abilities, language skills, hyperactivity levels, and other developmental f
 actors. Schools use these assessments to form balanced classes, ensuring that children with 
 varying skill levels and degrees of hyperactivity are evenly distributed. The goal is for each 
 class to have average skill and hyperactivity levels close to the overall average, promoting an 
 equitable learning environment across classes. Even though this principle seemed to have little effect on the 
 cognitive and non-cognitive abilities of children \citep{schneider2013school}, 
 it has a positive effect on the stress level of the teacher and therefore 
 on the learning environment. Additionally, the schools aim to put befriended children in the 
 same class as placing friends in the same class can have several beneficial 
 impacts on both social and academic outcomes. \citep{hallinan1978classroom}, \citepothers{Gail}
 \\
 
To create a class formation that, satisfies these goals and creates the desired environment for children and teachers can 
be challenging. Therefore, teachers spent a significant amount of time finding a fitting and 
 fair class distribution before the beginning of the school year. 

This complex challenge calls for innovative solutions that uses modern technology, 
data-driven decision-making, and heuristic approaches. Within this context, Evolutionary 
Algorithms emerge as a promising tool to navigate the intricacies of class allocation. 
These computational methods is abstracted from the principles of evolution, natural selection, 
and genetic variation. Evolutionary Algorithms are particularly useful when dealing with complex 
and non-linear optimization, constrain satisfaction problems or multi-objective optimization problems 
such as class allocation.
\\

In the context of this Applied Research Project for the Master's Degree Bionics 
(M. Sc.) a first implementation of an EA is created, which solves the class partitioning 
problem or presents a sufficiently adequate solution.

\subsection{Evolutionary Algorithms}
Evolutionary algorithms (EAs) are population-based stochastic search techniques inspired by 
the principles of biological evolution. They are designed to find approximate solutions to 
complex problems
\citep{bartz2014evolutionary}. Due to their flexibility and robustness, EAs excel at handling 
non-linear, high-dimensional, and poorly understood problems. These characteristics make them 
suitable for a wide range of applications. EAs offer a heuristic approach to optimization, making 
it possible to find satisfactory solutions in cases where optimal solutions either do not exist or 
are infeasible to compute using conventional methods. \citep{bartz2014evolutionary,michalewicz2013solve}

\citet{michalewicz2013solve} describes the important steps and principles of EA's as follows:


\begin{enumerate}
  \item \textbf{Creating an initial Population}: A population is a collection of individuals (or candidate solutions).
    Individuals represent potential solutions to the problem being addressed. Each individual contains a set of parameters or variables, 
    often referred to as its genotype, which encodes a solution. Individuals have the same genes, but different allels (variation of a gene).
    The initial populations exists of an amount of individuals of different allels.

  \item \textbf{Implementing a Fitness}: fitness refers to the measure of how well an individual 
   performs with respect to the optimization problem. The fitness function evaluates the quality of each individual in the population
   and is a mathematical or computational function.

  \item \textbf{Create an Evolutionary Loop}: The evolutionary loop is the iterative process in evolutionary algorithms 
  (EAs) that simulates the principles of natural evolution, such as selection, variation, and 
  reproduction. The loop continuously evolves a population of candidate solutions through multiple 
  generations until a stopping condition (e.g., a maximum number of generations, a fitness threshold, 
  or convergence) is met. The loop can be divided into different important aspects:
  \begin{enumerate}
    \item \textbf{Choose Parents}: Choose Parents is the process of choosing individuals from the population 
    to produce  offspring for the next generation. The aim of parent selection is to favor individuals 
    with higher fitness, allowing better solutions to pass their traits to the next generation while 
    maintaining diversity in the population to explore different areas of the solution space. There are
    different options for the selecting method such as Roulette Wheel Selection or Tournament Selection.
    \item \textbf{Create Offspring}:  Create Offspring involves applying genetic 
    operators to selected parent individuals to generate new solutions for the next generation. 
    The two primary genetic operators used for creating offspring are crossover and mutation.
    \item \textbf{Selection of the next Generation}: Selection of the next Generation is the step that 
    determines which individuals from the current population will be chosen to contribute 
    their genetic material to the next generation. The goal is to ensure that the more fit 
    individuals are preserved and allowed to reproduce, while also maintaining diversity 
    within the population to prevent converging to a local max and to explore different areas of the 
    solution space. There are different options for the selecting method such 
    as Roulette Wheel Selection or Tournament Selection.
  \end{enumerate}
\end{enumerate}





\subsection{Evolutionary Algorithms for multi-objective optimization Problems}
Multi-objective problems (MO) involve optimizing two or more conflicting objectives 
simultaneously. Unlike single-objective optimization, where a single goal
is pursued, multi-objective optimization requires balancing trade-offs among multiple criteria, sometimes 
conflicting each other. There are several possible approaches to deal with multi-objective problems, these are the most common ones
 \citep{bartz2014evolutionary}:
\begin{itemize}
  \item \textbf{Transfer to a single-objective problem}: combining multiple objectives into one comprehensive objective function. This 
  transformation allows optimization techniques focusing on single objectives to be applied. 
  \item \textbf{Pareto-Based Methods}: solving by on finding a set of solutions 
  that represent trade-offs between conflicting objectives. Pareto-based methods 
   aim to find the Pareto optimal set, which consists of solutions where no objective can be 
   improved without worsening at least one other objective. 
  \item \textbf{Lexicographical Approach }:  assigns varying levels of priority to different 
  objectives, focusing on optimizing them in a specific order based on their importance. 
  Higher-priority objectives are addressed first, and once they are optimized, attention shifts to
   the next objectives in the sequence.
\end{itemize}
Beyond these approaches, there are many other variations and hybrid methods. These are often 
specialized for particular use cases or aim to enhance specific aspects, such as maintaining 
diversity in a population, setting dominance relationships, weighting objectives, or solving 
sub-problems more effectively. \citep{ojha2019review}


\subsection{Aim of this Work}

In this project, an evolutionary algorithm is applied to address a class 
distribution problem. Although there are no clear and direct trade-offs in the problem, 
subtle, non-visible trade-offs may arise due to specific individuals or a small subset of 
individuals, especially in a small school setting. The key question is whether a Pareto-optimal 
method, which requires more computational power, is necessary for this problem, or if transforming
it into a single-objective problem would be sufficient.
\\

For the single-objective problem the Weighted-Sum-Method (WSM) is used, where the fitness value of each characteristic
of one candidate are given a weight and then summed up.

For the pareto-optimal method the Strength Pareto Evolutionary Algorithm 2 (SPEA2) created by 
\citet{zitzler2001spea2} is 
implemented. SPEA2 assigns fitness based on dominance (how many solutions a solution outperforms)
 and density 
(how crowded the space around a solution is), using the "k-th nearest neighbor" 
method to ensure diverse solutions. It selects parents from a combination of the main 
population and an archive of the best non-dominated solutions.
\\

It is hypothesized that a Pareto-optimal method, Strength Pareto Evolutionary Algorithm 2 
(SPEA2),
 will either yield better results or match 
the performance of the single-objective approach, Weighted-Sum-Method (WSM). The quality of the algorithm will be evaluated
 based on computational time, the number of generations 
required, and the average fitness of the individuals in the final population.


\section{Material and Methods}
In the following chapter the system requirements for a basic approach for this problem are specified and the implementation of the evolutionary algorithm is described.

\subsection{System Requirements}
The following functional requirements and restraints are defined for the first approach 
of the algorithm:

\begin{enumerate}

  \item \textbf{Restraints for class distribution}, class size, and number of students are  
  predefinitions and data from Ministry of Education for public primary schools of NRW as well as 
   from some public primary schools of NRW
  \citep{Quantita2020,Klassnebildung2020} \citepothers{rp-online}:
  
  \begin{itemize}
      \item The class size must be over 14 and must be under 30
      \item The minimal number of students is 31 the maximal number 600
      \item The minimal class number is 2 and the maximum 7
  \end{itemize}
  
  \item \textbf{Important characteristics for class distribution} of the children are taken from:
  
   \begin{itemize}
        \item The mean of males and females per class should be near the means of the overall students enrolling
        \item The mean and standard deviation of hyperactivity per class should be near the mean and standard deviation  of the overall students enrolling
        \item The mean and standard deviation  of language skill per class should be near the mean and standard deviation  of the overall students enrolling
        \item The mean and standard deviation  of logical skill per class should be near the mean and standard deviation  of the overall students enrolling
        \item It should be possible to enter a max. of three friends per student. 
    \end{itemize}
\end{enumerate}

The following non-functional requirements for the first basic version are defined:

\begin{enumerate}
  \item \textbf{Programming language} is C
  \item \textbf{The processing time} of the program does not exceed 60\,s.
\end{enumerate}


\subsection{Implementation}
The Evolutionary Algorithm to solve the problem was implemented is several separated steps.
\subsubsection{Creation of Candidates(Individuals)}
For creating a candidate (a class distribution) 
the first step is to implement the students themselves 
and their characteristics. The values for characteristics 
are shown in Table \ref{Tab:values}.


\begin{table} [H]
    \centering
    \caption[Students characteristics and their possible values.]{\textbf{Students characteristics and their possible values.}Very of the six characteristics has a number (index) in the code and a possible value.}

    \label{Tab:values} 
    
    \begin{tabular}{c|c|c|c}
        no. & characteristics  & possible values & value in code \\ \hline \hline
        0 & Student number & 0 to (no. of students - 1) &  0 to (no. of students - 1) \\\hline 

        1 & Sex & male, female &  0,1\\ \hline

        2 & Hyperactivity level & very low, low, average, high, very high & 0 to 4   \\ \hline 
        
        3 & Logical skill level & very low, low, average, high, very high & 0 to 4    \\ \hline 
        
        4 & Language skill level& very low, low, average, high, very high & 0 to 4    \\\hline 
        
        5-7 & Friendships & respective student no. & 0 to (no. of students - 1), 0 to (no. of students - 1), 0 to (no. of students - 1)    \\\hline 
        8 & Class & 0 to (no. of classes-1)& 0 to  (no. of classes-1) \\\hline 
        
    \end{tabular}
\end{table}
With these values a list of students was implemented with uniformly randomly chosen values for sex, hyperactivity, logical skill, languange skill, and friendships. (see Table \ref{Tab:8students})

\begin{table} [H]
    \centering
    \caption[Eight students with randomly chosen values]{\textbf{Eight students with randomly chosen values.} The eight students are respesentative for the 59 students in the program, the values are uniformly random chosen.}

    \label{Tab:8students} 

    
    \begin{tabular}{c|c|c|c|c|c}
        student no. & sex & hyperactivity  & log. skill & lang. skill & friendships \\ \hline \hline
        

        0 & female &  high & average & very high & 3\\ \hline

        1 &  male & very high & high & low & 4   \\ \hline 
        
        2 &  male & low & very high &  high &    \\ \hline 
        
        3 &  female & very low & high & very high & 0   \\ \hline 
        
        4 &  male & very high & average & very low & 1   \\ \hline  

        5 &  male & average & high & high & 6,7,2   \\ \hline
        
        6 &  female & very high & very low & average & 5,7   \\ \hline
        
        7 &  male & very low &very high & low & 5,6   \\ \hline
        
        
    \end{tabular}
\end{table}
After implementing the list of students,  a population size of 1000 candidates is created. To generate a matrix of students in a uniformly random order, a random permutation is generated for the numbers from 0 to 58, and each student is assigned to their respective number. 


To create classes, the number of students in each class is determined by generating a standard normal random variable \textit{N(19,67|1,56)}. The mean is the theoretical average of students per class. The standard deviation is calculated in a way that ensures the number of students per class does not exceed the allowed limits (15 and 29). To convert normally distributed values from uniformly random distributed values, the Box-Muller-Transformation is used. If the number of students per class exceeds the allowed limits, the values are recalculated to stay within the boundaries. 


\subsubsection{Implementation of fitness}
To compare the fitness values of candidates and to heuristically solve this multi-objective optimization problem, the difference between the mean of each class in one candidate and the mean of all students is calculated and summed up for the four characteristics: sex, hyperactivity, logical skill, and language skill. For every characteristic, the 
fitness values for one characteristic (sex, hyperactivity, logical skill level, and language skill level) are calculated as follows (Equation \ref{eq:fitness}):

\begin{equation}
    fit_{char} = \sum_{1}^{i}\sqrt{(\bar{x}_{students}-\bar{x}_{i})^2} + \sum_{1}^{i}\sqrt{(\sigma_{students}-\sigma_{i})^2}
    \label{eq:fitness}
\end{equation}

$fit_{char}$ is the fitness of a characteristic, $\bar{x}_{students}$ is the mean of the values of 
the respective characteristics of all students in one candidate, $\bar{x}_{i}$ is the mean 
of the values of the respective characteristics of class $i$ of the candidate, $\sigma_{students}$ is
 the standard deviation of the values of the respective characteristics of all students in one candidate, 
 and $\sigma_{i}$ is the standard deviation  of the
 values of the respective characteristics of class $i$ of the candidate. 

For calculating the fitness value of friendships for a candidate a maximal 
value was defined (3 times the number of students), which is the value of not fulfilling 
any friendships with the respective combination of students and classes. For every fulfilled 
friendship 1 is subtracted. 
\\

For every fitness value, the optimal is 0.
\\

As two different methods, transfer to a Weighted-Sum-Method (WSM) and Strength 
Pareto Evolutionary Algorithm 2 (SPEA2).

For the \textbf{Weighted-Sum-Method (WSM)} the fitness was calculated as follows:
\begin{equation}
  fit_{WSM} = \sum_{0}^{i}(w_i \cdot f_{char_i})
  \label{eq:WSM}
\end{equation}
Where  $f_{char_i}$ is the fitness of characteristic $i$ and $w_i$ is the weight put on the fitness.
As the different characteristics are of different scale and types the weights where assigned as follows:
\begin{itemize}
\item $w_i$ of $i = 1,2,3$ are 1
\item  $w_i$ of $i = 0$ is 4
\item $w_i$ of $i = 4$ is $\frac{number-of-students}{4}$
\end{itemize}
This method enables an equal influence of the characteristics on the fitness value.

For the \textbf{Strength Pareto Evolutionary Algorithm 2 (SPEA2)} the implementation of the fitness is 
taken from \citet{zitzler2001spea2}. The fitness consist of two values, the strength value and the density value.
The strength is calculated as follows:
\begin{equation}
  fit_{SPEA2_t} = \sum_{0}^{j}S_j
  \label{eq:WSM}
\end{equation}
Where the fitness $fit_{SPEA2_t}$ for one individual t is the sum of strength values $S_j 	\in [0; 1) $.
strength values $S_j$ is assigned by comparing each $fit_{char_i}$ of the respective individual t with 
the $fit_{char_i}$ of every other individual j. 


\[
    S_j =
\begin{cases}
 0${, if individual t is dominated by or has equal values as individual j in all characteristic}$\\
  1${, if individual t dominates  individual j in one characteristics}$
\end{cases}
\]
The density of each individual t is calculated using the k-nearest neighbor method. Therefore, 
the  Euclidean distances to every other individual j is calculated: 

\begin{equation}
  d_{t,j} = \sqrt{\sum_{0}^{i}(fit_{char_i,t}-fit_{char_i,j})^2}
  \label{eq:distance}
\end{equation}
Where $fit_{char_i,t}$ is the the fitness of individual t for the characteristics i and  $fit_{char_i,j}$ is the fitness of individual j for the
characteristics i. 

The distance of individual j to every other individual is calculated and then sorted in
 list from smallest to largest distance. To determine the k-th neighbor $k$ is calculated as follows


 \begin{equation}
  k = \sqrt{N+ N_a}
  \label{eq:k}
\end{equation}

$N$ is the population size and $N_a$ is the size of the population in the archive.
The distance $d_{t,j}$ at the position $k$ of the list is then used to calculate the density $D_t$:

\begin{equation}
  D_t = \frac{1}{d_{t,j}+2}
  \label{eq:k}
\end{equation}
The overall fitness of individual t is the sum of the fitness value  $fit_{SPEA2_t}$ and the 
density value $D_t$ is then calculated as follows:


\begin{equation}
  F_t = fit_{SPEA2_t} + D_t
  \label{eq:k}
\end{equation}


 \subsubsection{Reproduction}


For selecting parents a tournament selection after horn nafpliotis was implemented. Therefore, two parents as well as a comparison set of 10 individuals were uniformly randomly selected from the population. The parents dominating (Pareto-based) more of the individuals in the comparison set are selected with a chance of 95\% to procreate. 

After the selection of a parent, point mutations are used to alter the permutation of students in the parent and to get a child with slightly different fitness values. If the child is dominating the parent it has a 95\% of being part of the next generation (replacing the parent) and vice versa.

The selection and reproduction process is repeated 10 times. 

\subsubsection{Evolution}
The process of reproduction is repeated for 100 generations 










\section{Results}













\section{Discussion}


\section{Conclusion}

  
\end{onehalfspace}


\newpage
\section*{References}
\label{sec:refs}

\renewcommand{\bibsection}{\subsection*{Literature}}
\label{sec:Literature}
\bibliography{References/Literature.bib}
\newpage

\renewcommand{\bibsection}{\subsection{Other Sources}}
%\label{sec:sonstigeQuellen}
\bibliographyothers{References/others.bib}

%\renewcommand{\bibsection}{\subsection{Patents and Norms}}
%\label{sec:PatentsNorms}
%\bibliographypatents{References/PatentsNorms.bib}



% Content: Team’s discretion
% Compliance matrix, whether requirements were met
+

% OVERALL Assesment Criteria: Quality of the Report, 15 points
% Total points: 150

%%%%%%%%%%%%%%%%%%%%%%% Literature %%%%%%%%%%%%%%%%%%%%%%


%%%%%%%%%%%%%%%%%%%%% Appendix %%%%%%%%%%%%%%%%%%%%%%%%%%%



\end{document}